%%%%%%%%%%%%%%%%%%%%%%%%%%%%%%%%%%%%%%%%%%%%%%%%%%%
%% LaTeX book template                           %%
%% Author:  Amber Jain (http://amberj.devio.us/) %%
%% License: ISC license                          %%
%%%%%%%%%%%%%%%%%%%%%%%%%%%%%%%%%%%%%%%%%%%%%%%%%%%

\documentclass[a4paper,11pt]{book}
\usepackage[colorlinks=true, linkcolor=blue]{hyperref}
% \usepackage[T1]{fontenc}
% \usepackage[utf8]{inputenc}
% \usepackage{lmodern}
%%%%%%%%%%%%%%%%%%%%%%%%%%%%%%%%%%%%%%%%%%%%%%%%%%%%%%%%%
% Source: http://en.wikibooks.org/wiki/LaTeX/Hyperlinks %
%%%%%%%%%%%%%%%%%%%%%%%%%%%%%%%%%%%%%%%%%%%%%%%%%%%%%%%%%
\usepackage{lindrew}
\usepackage{adjustbox}
%%%%%%%%%%%%%%%%%%%%%%%%%%%%%%%%%%%%%%%%%%%%%%%%
% Chapter quote at the start of chapter        %
% Source: http://tex.stackexchange.com/a/53380 %
%%%%%%%%%%%%%%%%%%%%%%%%%%%%%%%%%%%%%%%%%%%%%%%%

\makeatletter

\renewcommand{\@chapapp}{}% Not necessary...

\newenvironment{chapquote}[2][2em]
{\setlength{\@tempdima}{#1}%
	\def\chapquote@author{#2}%
	\parshape 1 \@tempdima \dimexpr\textwidth-2\@tempdima\relax%
	\itshape}
{\par\normalfont\hfill--\ \chapquote@author\hspace*{\@tempdima}\par\bigskip}
\makeatother

%%%%%%%%%%%%%%%%%%%%%%%%%%%%%%%%%%%%%%%%%%%%%%%%%%%
% First page of book which contains 'stuff' like: %
%  - Book title, subtitle                         %
%  - Book author name                             %
%%%%%%%%%%%%%%%%%%%%%%%%%%%%%%%%%%%%%%%%%%%%%%%%%%%

% Book's title and subtitle
\title{\Huge \textbf{Abstract Algebra}}
% Author
\author{\textsc{Hechen Hu}}

\begin{document}
	% \frontmatter
	\maketitle
	%%%%%%%%%%%%%%%%%%%%%%%%%%%%%%%%%%%%%%%%%%%%%%%%%%%%%%%%%%%%%%%%%%%%%%%%
	% Auto-generated table of contents, list of figures and list of tables %
	%%%%%%%%%%%%%%%%%%%%%%%%%%%%%%%%%%%%%%%%%%%%%%%%%%%%%%%%%%%%%%%%%%%%%%%%
	
	
	
	% \mainmatter
	

\tableofcontents

\include{1_Groups/groups}
\chapter{The Structure of Groups}
\section{Free Abelian Groups}
\section{Finitely Generated Abelian Groups}
\section{The Krull-Schmidt theorem}
\section{The Action of a Group on a Set}
\begin{definition}
	An \textit{action} of a group $ G $ on a set $ S $ is a function $ G \times S \to S $(usually denoted by $ (g,x)\mapsto gx $) such that for all $ x\in S $ and $ g_1,g_2\in G $:
	\begin{enumerate}
		\item $ ex=x $;
		\item $ (g_1g_2)x=g_1(g_2 x) $.
	\end{enumerate}
When such an action is given, $ G $ \textit{acts on the set $ S $}.
\end{definition}
\begin{definition}
	Let $ G $ be a group and $ H $ a subgroup. An action of the group $ H $ on the set $ G $ is given by $ (h,x)\mapsto hx $, where $ hx $ is the product in $ G $. The action of $ h\in H $ on $ G $ is called a \textit{(left) translation}.
\end{definition}
\begin{definition}
		Let $ G $ be a group and $ H $ a subgroup. An action of $ H $ on the set $ G $, given by $ (h,x)\mapsto hxh^{-1} $, is called \textit{conjugation by $ h $} and $ hxh^{-1} $ is said to be a \textit{conjugate of $ x $}. $ H $ can also act on the set $ S $ of all subgroups of $ G $ by conjugation $ (h,K)\mapsto hKh^{-1} $. The group $ hKh^{-1} $ is said to be \textit{conjugate to $ K $}.
\end{definition}
\begin{theorem}
	Let $ G $ be a group that acts on a set $ S $.
	\begin{enumerate}
		\item The relation on $ S $ defined by
		\begin{equation}
			x \sim x^{\prime}\Leftrightarrow gx=x^{\prime}\qquad \text{for some}\qquad g\in G\nonumber
		\end{equation}
		is an equivalence relation.
		\item For each $ x \in S $, $ G_x=\{g\in G|gx=x \} $ is a subgroup of $ G $.
	\end{enumerate}
\end{theorem}
\begin{definition}
	The equivalence classes of the equivalence relation previously mentioned are called the \textit{orbits} of $ G $ on $ S $; the orbit of $ x \in S $ is denoted $ \bar{x} $. The subgroup $ G_x$ is called the \textit{subgroup fixing $ x $}, the \textit{isotropy group of $ x $}, or the \textit{stabilizer of $ x $}.
\end{definition}
\begin{lemma}
	If $ G $ acts on $ S $, then for any $ x\in S $
	\begin{equation}
		|[G:\Stab x]|=| \Orb x|\nonumber
	\end{equation}
\end{lemma}
\begin{theorem}
		If $ G $ acts on $ S $, then for any $ x\in S $
	\begin{equation}
	|G|=|\Stab x| |\Orb x|\nonumber
	\end{equation}
\end{theorem}


\section{The Sylow theorem}
\section{Classification of Finite Groups}
\section{Nilpotent and Solvable Groups}
\section{Normal and Subnormal Series}


\chapter{Rings}
\section{Rings and Homomorphisms}
\begin{definition}
	A \textit{ring} is a nonempty set $ R $ together with two binary operations (usually denoted as addition $ (+) $ and multiplication) such that:
	\begin{enumerate}
		\item $ (R,+) $ is an abelian group;
		\item the multiplication is associative;
		\item $ a(b+c)=ab+ac $ and $ (a+b)c = ac + bc $ (left and right distributive laws).
	\end{enumerate}
If in addition the multiplication is commutative, $ R $ is said to be a \textit{commutative ring}. If $ R $ contains an identity element for multiplication, $ R $ is said to be a \textit{ring with identity}.
\end{definition}
The additive identity of the ring is called the zero element and denoted $ 0 $.
\begin{theorem}
	Let $ R $ be a ring. Then
	\begin{enumerate}
		\item $ 0a = a0 = 0 $ for all $ a \in R $;
		\item $ (-a)b=a(-b)=-(ab) $ for all $ a,b \in R$;
		\item $ -(a)(-b)=ab $ for all $ a,b \in R $;
		\item $ (na)b=a(nb)=n(ab) $ for all $ n \in Z $ and all $ a,b \in R $;
		\item \begin{equation}
		(\sum_{i=1}^{n}a_i)(\sum_{j=1}^{m}b_j) = \sum_{i=1}^{n} \sum_{j=i}^{m}a_i b_j \qquad \text{for all}\qquad a_i,b_j \in R \nonumber
		\end{equation}
	\end{enumerate}
\end{theorem}
\begin{definition}
	A nonzero element $ a $ in a ring $ R $ is said to be \textit{left} (resp. \textit{right}) \textit{zero divisor} if there exists a nonzero $ b \in R$ such that $ ab=0 $ (resp. $ ba=0 $). A \textit{zero divisor} is an element of $ R $ which is both a left and a right zero divisor.
\end{definition}
A ring has no zero divisors iff the right and left cancellation laws hold in this ring.
\begin{definition}
	An element $ a $ in a ring $ R $ with identity is said to be \textit{left} (resp. \textit{right}) \textit{invertible} if there exists $ c \in R $ (resp. $ b \in R $) such that $ ca= 1_R $ (resp. $ ab = 1_R $). The element $ c $ (resp. $ b $) is called a \textit{left} (resp. \textit{right}) \textit{inverse} of $ a $. An element $ a \in R $ that is both left and right invertible is said to be \textit{invertible} or to be a \textit{unit}.
\end{definition}
A unit's left and right inverses necessarily coincide. The set of units in a ring $ R $ with identity forms a group under multiplication.
\begin{definition}
	A commutative ring $ R $ with identity $ 1_R \neq 0 $ and no zero divisors is called an \textit{integral domain}. A ring $ D $ with identity $ 1_D \neq 0 $ in which every nonzero element is a unit is called a \textit{division ring}. A \textit{field is a commutative division ring}.
\end{definition}
\begin{theorem}[Binomial theorem]
	Let $ R $ be a ring with identity, $ n $ a positive integer, and $ a,b, a_1,a_2,\cdots, a_s \in R $.
	\begin{enumerate}
		\item If $ ab=ba $, then
		\begin{equation}
			(a+b)^n = \sum_{k=0}^{n}\binom{n}{k}a^k b^{n-k}  \nonumber
		\end{equation}
		\item If $ a_i a_j = a_j a_i $ for all $ i $ and $ j $, then
		\begin{equation}
			(a_1+a_2+\cdots +a_s)^n = \sum \frac{n!}{(i_1!)\cdots (i_s!)}a_1^{i_1}  a_2^{i_2} \cdots a_s^{i_s} \nonumber
		\end{equation}
		where the sum is over all s-tuples $ (i_1,i_2,\cdots,i_s) $ such that $ i_1 + i_2 + \cdots +i_s = n $.
	\end{enumerate}
\end{theorem}
\begin{definition}
	A \textit{homomorphism of rings} $ f :R \to S$ between two rings $ R $ and $ S $ is a mapping that preserves the ring structure, that is
	\begin{equation}
		f(r_1)f(r_2)=f(r_1 r_2) \qquad \text{and}\qquad f(r_1 + r_2)= f(r_1)+f(r_2) \nonumber
	\end{equation}
	for all $ r_1, r_2 \in R $. Because of its similarity with respect to the homomorphisms of groups, the same terminology (like monomorphisms, epimorphisms and isomorphisms for injective, surjective and bijective homomorphisms respectively) will also apply. A monomorphism of rings $ R \to S $ is sometimes called an \textit{embedding of $ R $ in $ S $}. The \textit{kernel} and \textit{image} of homomorphisms of rings are defined similar to those of group homomorphisms -- the only difference is that the homomorphism maps the elements in its kernel to the identity element $ 0 $ of the additive abelian group. \textbf{In fact if $ R $ and $ S $ both have identities $ 1_R $ and $ 1_S $ it is not required that a homomorphism maps $ 1_R $ to $ 1_S $.}
\end{definition}
\begin{Example}
	The canonical map $ \Zahlen \to \Zahlen_m $ defined by $ k \mapsto \bar{k} $ is an epimorphism of rings.
\end{Example}
\begin{definition}
	Let $ R $ be a ring. If there is a least positive integer $ n $ such that $ na=0 $ for all $ a \in R $, then $ R $ is said to \textit{have characteristic $ n $}. If no such $ n $ exists $ R $ is said to \textit{have characteristic $ 0 $}.(Notation: $ \Char R = n $)
\end{definition}
\begin{theorem}
	Let $ R $ be a ring with identity $ 1_R $ and characteristic $ n > 0 $.
	\begin{enumerate}
		\item If $ \varphi:Z \to R $ is the map given by $ m \mapsto m 1_R $, then $ \varphi $ is a homomorphism with kernel $ \angbracket{n} $.
		\item $ n $ is the least positive integer such that $ n 1_r = 0 $.
		\item If $ R $ has no zero divisors($ R $ is an integral domain), then $ n $ is prime.
	\end{enumerate}
\end{theorem}
\begin{theorem}
	Every ring $ R $ may be embedded in a ring $ S $ with identity. The ring $ S $ (which is not unique) may be chosen to be either of characteristic zero or of the same characteristic as $ R $.
\end{theorem}
\begin{definition}
	A ring $ R $ such that $ a^2 =a  $ for all $ a \in R $ is called a \textit{Boolean Ring}. Every Boolean ring is commutative and $ a+a =0 $ for all $ a \in R $.
\end{definition}
\begin{theorem}[a.k.a The Freshman's Dream]
	If $ R $ is a commutative ring with identity of prime characteristic $ p $ and $ a, b \in R $, then $ (a \pm b)^{p^n}=a^{p^n}\pm b^{p^n} $ for all $ n \geqslant 0 \in \Zahlen $ (note that $ b = -b $ if $ p=2 $).
\end{theorem}
\begin{definition}
	An element $ a $ of a ring is \textit{nilpotent} if $ a^n=0 $ for some integer $ n $.
\end{definition}
\begin{theorem}
	In a commutative ring $ a+b $ is nilpotent if $ a $ and $ b $ are.
\end{theorem}
However, the theorem is not necessarily true in a non-commutative ring. For example, in the ring over all $ 2 \times 2 $ matrices over $ \Real $ where addition and multiplication are defined respectively by matrix addition and multiplication the elements $ \begin{pmatrix}
0 &1 \\
0 &0
\end{pmatrix} $ and $ \begin{pmatrix}
0 &0 \\
1 &0
\end{pmatrix} $ are nilpotent(their square equals to the additive zero in this ring), but their sum is not.
\begin{theorem}
	A finite ring with more than one element and no zero divisors is a division ring.
\end{theorem}
\begin{proof}
	For each non-zero element $ a \in R $ define the map $ \varphi_a : R \to R $ given by $ x \mapsto ax (x \in R)$. Show that the map is a bijection and thus an identity exists as well as $ a $ is invertible.
\end{proof}
\begin{definition}
	The homomorphism $ R \to R $ defined on a commutative ring $ R $ with identity and prime characteristic $ p $ given by $ r \mapsto r^p $ is called the \textit{Frobenius homomorphism}.
\end{definition}
\begin{definition}
	If $ R $ is a ring, then so is $ R^{op} $, where $ R^{op} $ is defined as follows: their underlying set is the same; their addition coincide; the multiplication in $ R^{op} $ is defined by $ a \circ b = ba $, where $ ba $ is the product in $ R $. The ring $ R^{op} $ is called the \textit{opposite ring} of $ R $.
\end{definition}
\begin{theorem}
	If $ R $ and $ S $ are rings and $ R^{op} $ and $ S^{op} $ are their respective opposite rings, then
	\begin{enumerate}
	\item	$ R $ has an identity iff $ R^{op} $ does;
	\item $ R $ is a division ring iff $ R^{op} $ is;
	\item $ (R^{op})^{op}=R $;
	\item If $ S $ is a ring, then $ R \cong S $ iff $ R^{op}\cong S^{op} $.
	\end{enumerate}
\end{theorem}





\section{Ideals}
\begin{definition}
	Let $ R $ be a ring and $ S $ a nonempty subset of $ R $ that is closed under addition and multiplication in $ R $. If $ S $ is itself a ring under these operations then $ S $ is called a \textit{subring} of $ R $. A subring $ I $ of $ R $ is a \textit{left ideal}(resp. \textit{right ideal}) provided for $ r \in R $ and $ x \in I $ we have $ rx \in I $(resp. $ xr\in I $). $ I $ is an \textit{ideal} if it is both a left and right ideal.
\end{definition}
It can be seen that ideal is the analogous definition of a normal subgroup of a group.
\begin{Example}
	The \textit{center} of a ring $ R $ is the set $ C = \{c \in R | cr=rc \text{ for all }r \in R\} $. $ C $ is a subring of $ R $ but it may not be an ideal.
\end{Example}
\begin{Example}
	The cyclic group generated by any integer $ n $ is an ideal in $ \Zahlen $.
\end{Example}
\begin{definition}
	The ideal of a ring that only contains $ 0 $ is called the \textit{trivial ideal}(denoted $ 0 $). An ideal $ I $ of $ R $ such that $ I $ is not trivial and $ I \neq R $ is called a \textit{proper ideal}.
\end{definition}
If $ R $ has an identity $ 1_R $ and $ I $ is an ideal of $ R $, then $ I=R $ iff $ 1_R \in I $. Consequently a nonzero ideal $ I $ is proper iff $ I $ contains no units of $ R $. In particular, a division ring has no proper ideals.
\begin{theorem}
	A nonempty subset $ I $ of $ R $ is a left (resp. right) ideal iff for all $ a,b \in I $ and $ r \in R $:
	\begin{enumerate}
		\item $ a,b \in I \Rightarrow a-b\in I $;
		\item $ a \in I, r \in R \Rightarrow ra \in I $(resp. $ ar \in I $).
	\end{enumerate}
\end{theorem}
\begin{Corollary}
	Let $ \{A_i | i \in I \} $ be a family of [left] ideals in a ring $ R $. Then $ \bigcap_{i \in I}A_i $ is also a [left] ideal.
\end{Corollary}
\begin{definition}
	Let $ X $ be a subset of a ring $ R $. Let $ \{A_i | i \in I \} $ be the family of all [left] ideals in $ R $ which contain $ X $. Then $ \bigcap_{i \in I} A_i $ is called the [left] \textit{ideal generated by $ X $}. This ideal is denoted $ (X) $. The elements of $ X $ are called \textit{generators} of $ (X) $. If $ X $ is finite, then $ (X) $ is said to be \textit{finitely generated}. An ideal $ (x) $ generated by a single element is called a \textit{principal ideal}. A \textit{principle ideal ring} is a ring in which every ideal is principal. A principal ideal ring which is an integral domain is called a \textit{principal ideal domain}.
\end{definition}
\begin{theorem}
	Let $ R $ be a ring, $ a \in R $ and $ X \subset R $.
	\begin{enumerate}
	\item The principal ideal $ a $ consists of all elements of the form $ ra + as+ na + \sum_{i=1}^{m}r_i a s_i $($ r,s,r_i,s_i\in R $; $ m \in \Natural $; and $ n \in \Zahlen $).
	\item If $ R $ has an identity, then $ (a)=\{\sum_{i=1}^{n}r_i a s_i |r_i, s_i \in R; n \in \Natural \} $.
	\item If $ a $ is in the center of $ R $, then $ (a)=\{ra+na|r \in R, n\in Z \} $.
	\item $ Ra = \{ra|r \in R \} $(resp. $ aR = \{ar|r \in R \} $) is a left(resp. right) ideal in $ R $(which may not contain $ a $). If $ R $ has an identity, then $ a \in Ra $ and $ a \in aR $.
	\item If $ R $ has an identity and $ a $ is in the center of $ R $, then $ Ra = (a)=aR $.
	\item If $ R $ has an identity and $ X $ is in the center of $ R $, then the ideal $ (X) $ consists of all finite sums $ r_1 a_1 + \cdots + r_n a_n(n \in \Natural; r_i \in R; a_i \in X)$.
	\end{enumerate}
\end{theorem}
\begin{definition}
	Let $ A_1,A_2,\cdots,A_n $ be nonempty subsets of a ring $ R $. Denote by $ A_1 + A_2 + \cdots + A_n $ the set $ \{a_1+a_2+\cdots +a_n|a_i \in A_i\text{ for all } i \} $. If $ A $ and $ B $ are nonempty subsets of $ R $ let $ AB $ denote the set of all finite sums $ \{a_1 b_1 + \cdots +a_n b_n |n\in \Natural, a_i \in A, b_i \in B \} $. The definition of $ AB $ can be extended to an arbitrary number of factors. If all factors are the same set $ A $ it is denoted by $ A^n $.
\end{definition}
\begin{theorem}
	Let $ A,A_1,A_2,\cdots, A_n $, $ B $ and $ C $ be [left] ideals in a ring $ R $.
	\begin{enumerate}
		\item $ A_1 + A_2 + \cdots + A_n $ and $ A_1  A_2  \cdots A_n  $ are [left] ideals;
		\item $ (A+B)+C = A+(B+C) $;
		\item $ (AB)C = A(BC)=ABC $;
		\item $ B(A_1+A_2+\cdots + A_n) =BA_1+BA_2+\cdots + BA_n $ and $ (A_1+A_2+\cdots + A_n)C = A_1 C+A_2 C+\cdots + A_n C $(distributivity).
	\end{enumerate}
\end{theorem}
Since $ R $ is additively abelian, any ideal of it is also a normal subgroup. Thus the quotient $ R/I $ group can be defined in which addition is given by $ (a+I)+(b+I) =(a+b)+I$. Moreover, $ R/I $ can be made into a ring.
\begin{theorem}
	Let $ R $ be a ring and $ I $ an ideal of $ R $. Then the additive quotient group $ R/I $ is a ring with multiplication given by
	\begin{equation}
		(a+I)(b+I)=(ab+I) \nonumber
	\end{equation}
	If $ R $ is commutative or has an identity, then the same is true of $ R/I $.
\end{theorem}
\begin{theorem}
	If $ f:R \to S $ is a homomorphism of rings, then the kernel of $ f $ is an ideal in $ R $. Conversely if $ I $ is an ideal in $ R $, then the map $ \pi:R \to R/I $ given by $ r \mapsto r +I $ is an epimorphism of rings with kernel $ I $.
\end{theorem}
The map $ \pi $ is called the \textit{canonical epimorphism}(or \textit{projection}).
\begin{theorem}
	If $ f:R \to S $ is a homomorphism of rings and $ I $ is an ideal of $ R $ which is contained in the kernel of $ f $, then there is a unique homomorphism of rings $ \bar{f}:R/I \to S $ such that $ \bar{f}(a+I)=f(a) $ for all $ a \in R $. $ \Im \bar{f}=\Im f $ and $ \Ker \bar{f}=(\Ker f)/I $. $ \bar{f} $ is an isomorphism iff $ f$ is an epimorphism and $ I = \Ker f $.
\end{theorem}
\begin{Corollary}[First Isomorphism theorem]
	If $ f:R \to S $ is a homomorphism of rings, then $ f $ induces an isomorphism of rings $ R/\Ker f \cong \Im f $.
\end{Corollary}
\begin{Corollary}
	If $ f:R \to S $ is a homomorphism of rings, $ I $ is an ideal in $ R $ and $ J $ is an ideal in $ S $ such that $ f(I)\subset J $, then $ f $ induces a homomorphism of rings $ \bar{f}:R/I \to S/J $, given by $ a+I \mapsto f(a)+J $. $ \bar{f} $ is an isomorphism iff $ \Im f +J = S $ and $ f^{-1}(J)\subset I $. In particular, if $ f $ is an epimorphism such that $ f(I)=J $ and $ \Ker f \subset I $, then $ \bar{f} $ is an isomorphism.
\end{Corollary}
\begin{theorem}
	Let $ I $ and $ J $ be ideals in a ring $ R $.
	\begin{enumerate}
		\item (Second Isomorphism theorem) There is an isomorphisms of rings $ I/(I \cap J)\cong (I+J)/J $;
		\item (Third Isomorphism theorem) if $ I \subset J $, then $ J/I $ is an ideal in $ R/I $ and there is an isomorphism of rings $ (R/I)/(J/I)\cong R/J $.
	\end{enumerate}
\end{theorem}
\begin{theorem}
	If $ I $ is an ideal in a ring $ R $, then there is a bijection between the set of all ideals of $ R $ which contain $ I $ and the set of all ideas of $ R/I $, given by $ J \mapsto J/I $. Hence every ideal in $ R/I $ is of the form $ J/I $, where $ J $ is an ideal of $ R $ which contains $ I $.
\end{theorem}
\begin{definition}
	An ideal $ P $ in a ring $ R $ is said to be \textit{prime} if $ P \neq R $ and for any ideals $ A,B  $ in $ R $
	\begin{equation}
		AB \subset P \quad\Rightarrow\quad  A \subset P \quad \text{ or }\quad B \subset P \nonumber
	\end{equation}
\end{definition}
\begin{theorem}
	If $ P $ is an ideal in a ring $ R $ such that $ P \neq R $ and for all $ a,b \in R $
	\begin{equation}
		ab \in P \quad \Rightarrow \quad a \in P \quad \text{or} \quad b \in P \nonumber
	\end{equation}
	then $ P $ is prime. Conversely if $ P $ is prime and $ R $ is commutative, then $ P $ satisfies the condition above.
\end{theorem}
\begin{Example}
	The zero ideal in any integral domain is prime. If $ p $ is a prime integer, then the principal ideal $ (p) $ in $ \Zahlen $ is prime.
\end{Example}
\begin{theorem}
	In a commutative ring $ R $ with identity $ 1_R \neq 0 $ and ideal $ P $ is prime iff the quotient ring $ R/P $ is an integral domain.
\end{theorem}
\begin{definition}
	An ideal [resp. left ideal] $ M $ in a ring $ R $ is said to be \textit{maximal} if $ M \neq R $ and for every ideal [resp. left ideal] $ N $ such that $ M \subset  N \subset R$, either $ N = M $ or $ N = R $.
\end{definition}
If $ R $ is a ring and $ \mathscr{S} $ is the set of all ideals $ I $ of $ R $ such that $ I \neq R $, then $ \mathscr{S} $ is partially ordered by set-theoretic inclusion. Consequently the following theorem can be proved using Zorn's lemma.
\begin{theorem}
	In a nonzero ring $ R $ with identity maximal [left] ideals always exist. In fact every [left] ideal in $ R $ except $ R $ itself is contained in a maximal [left] ideal.
\end{theorem}
\begin{theorem}
	If $ R $ is a commutative ring such that $ R^2 = R $(in particularly if $ R $ has an identity), then every maximal ideal $ M $ in $ R $ is prime.
\end{theorem}
\begin{theorem}
	Let $ M $ be an ideal in a ring $ R $ with identity $ 1_R \neq 0 $.
	\begin{enumerate}
	\item If $ M $ is maximal and $ R $ is commutative, then the quotient ring $ R/M $ is a field.
	\item If the quotient ring $ R/M $ is a division ring, then $ M $ is maximal.
	\end{enumerate}
\end{theorem}
\begin{Corollary}
	The following conditions on a commutative ring $ R $ with identity $ 1_R \neq 0 $ are equivalent.
	\begin{enumerate}
		\item $ R $ is a field;
		\item $ R $ has no proper ideals;
		\item $ 0 $ is a maximal ideal in $ R $;
		\item every nonzero homomorphism of rings $ R \to S $ is a monomorphism.
	\end{enumerate}
\end{Corollary}
\begin{theorem}
	Let $ \{R_i | i\in I \} $ be a nonempty family of rings and $ \prod_{i \in I} R_i $ the direct product of the additive abelian group $ R_i $;
	\begin{enumerate}
		\item $ \prod_{i \in I}R_i $ is a ring with multiplication defined by $ \{a_i \}_{i \in I} \{a_i \}_{i \in I} = \{a_i b_i\}_{i \in I} $;
		\item if $ R_i $ has an identity[resp. is commutative] for every $ i \in I $, then $ \prod_{i \in I} R_i $ has an identity [resp. is commutative];
		\item for each $ k \in I $ the canonical projection $ \pi_k: \prod_{i \in I} R_i \to R_k $ given by $ \{a_i \}\mapsto a_k $ is an epimorphism of rings;
		\item for each $ k \in I $ the canonical injection $ \iota_k: R_k \to \prod_{i \in I} R_i $ given by $ a_k \mapsto \{a_i \} $(where $ a_i = 0 $ for $ i \neq k $) is a monomorphism of rings.
	\end{enumerate}
\end{theorem}
\begin{definition}
	$ \prod_{i \in I} R_i $ is called the \textit{(external) direct product} of the family of rings. Its notation is analogous with it of the direct product of groups.
\end{definition}
\begin{theorem}
	$ \prod_{i \in I} R_i$ is a product in the category of rings.
\end{theorem}
\begin{theorem}
	Let $ A_1, A_2,\cdots,A_n $ be ideals in a ring $ R $ such that
	\begin{enumerate}
		\item $ A_1 + A_2 + \cdots + A_n = R $;
		\item for each $ k(1 \leqslant k \leqslant n) $, $ A_k \cap (A_1 + \cdots + A_{k-1}+A_{k+1}+\cdots + A_n)=0 $.
	\end{enumerate}
Then there is a ring isomorphism $ R \cong A_1 \times A_2 \times \cdots \times A_n $.
\end{theorem}
If a ring and a family of its ideals satisfies the conditions in the theorem above, the ring is said to be the \textit{(internal) direct product} of this family of ideals. The notation of (internal) direct product for a ring is analogous to it of (internal) direct product for a group.
\begin{definition}
	Let $ A $ be an ideal in a ring $ R $ and $ a,b\in R $. The element $ a $ is said to be \textit{congruent to $ b $ modulo $ A $}(denoted $ a\equiv b \pmod{A} $) if $ a-b \in A $. Thus
	\begin{equation}
		a \equiv b \pmod{A} \Leftrightarrow a-b \in A \Leftrightarrow a+A = b+A \nonumber
	\end{equation}
\end{definition}
\begin{theorem}[Chinese Remainder theorem]
	Let $ A_1,\cdots, A_n $ be ideals in a ring $ R $ such that $ R^2 + A_i =R$ for all $ i $ and $ A_i + A_j = R $ for all $ i \neq j $. If $ b_1,\cdots,b_n \in R $, then there exists $ b\in R $ such that
	\begin{equation}
		b \equiv b_i \pmod{A_i}\qquad (i=1,2,\cdots,n) \nonumber
	\end{equation}
	Furthermore $ b $ is uniquely determined up to congruence modulo the ideal
	\begin{equation}
		A_1 \cap A_2 \cap \cdots \cap A_n \nonumber
	\end{equation}
\end{theorem}
\begin{Corollary}
	Let $ m_1,\cdots,m_n $ be positive integers such that $ (m_i,m_j)=1 $ for $ i \neq j $. If $ b_1, b_2,\cdots, b_n $ are any integers, then the system of congruences
	\begin{equation}
		x \equiv b_1 \pmod{m_1}; x \equiv b_2 \pmod{m_2};\cdots;x \equiv b_n \pmod{m_n} \nonumber
	\end{equation}
	has an integral solution that is uniquely determined up to modulo $ m = m_1 m_2 \cdots m_n $.
\end{Corollary}
\begin{Corollary}
	If $ A_1,A_2,\cdots,A_n $ are ideals in a ring $ R $, then there is a monomorphism of rings
	\begin{equation}
		\theta: R/(A_1 \cap \cdots \cap A_n)\to R/A_1 \times R/A_2 \times \cdots \times R/A_n \nonumber
	\end{equation}
	If $ R^2 + A_i = R $ for all $ i $ and $ A_i + A_j = R $ for all $ i \neq j $, then $ \theta $ is an isomorphism of rings.
\end{Corollary}
\begin{theorem}
	The equation $ ax+ny=b $ has solutions for $ x,y \in \Zahlen $ iff $ GCD(a,n)|b $.
\end{theorem}
\begin{theorem}
	The congruence $ ax \equiv b \mod n $ has a solution iff $ GCD(a,n)|b $. Moreover, if this congruence does have at least one solution, the number of noncongruent solutions modulo $ n $ is $ GCD(a,n) $; that is, if $ [a][x]=[b] $ has a solution in $ \Zahlen_n $, then it has $ GCD(a,n) $ different solutions in $ \Zahlen_n $.
\end{theorem}
\begin{definition}
	Let $ m = m_1 m_2 \cdots m_r $, where the integers $ m_i $ are coprime in pairs. The \textit{residue representation} or \textit{modular representation} of any number $ x $ in $ \Zahlen_m $ is the $ r $-tuple $ (a_1,a_2,\cdots,a_r) $ where $ x\equiv a_i \mod m_i $.
\end{definition}


\section{Factorization in Commutative Rings}
\begin{definition}
	A nonzero element $ a $ of a commutative ring $ R $ is said to \textit{divide} an element $ b $ in $ R $(written $ a|b $) if there exists $ x \in R $ such that $ ax=b $. Elements $ a,b $ of $ R $ are said to be \textit{associates} if $ a|b $ and $ b|a $.
\end{definition}
\begin{theorem}
	Let $ a,b $ and $ u $ be elements of a commutative ring $ R $ with identity.
	\begin{enumerate}
		\item $ a|b $ iff $ (b)\subset (a) $.
		\item $ a $ and $ b $ are associates iff $ (a)=(b) $.
		\item $ u $ is a unit iff $ u|r $ for all $ r \in R $.
		\item $ u $ is a unit iff $ (u)=R $.
		\item The relation ``$ a $ is an associate of $ b $'' is an equivalence relation on $ R $.
		\item If $ a=br $ with $ r \in R $ a unit, then $ a $ and $ b $ are associates. If $ R $ is an integral domain, the converse is true.
	\end{enumerate}
\end{theorem}
\begin{definition}
	Let $ R $ be a commutative ring with identity. An element $ c $ of $ R $ is \textit{irreducible} provided that
	\begin{enumerate}
		\item $ c $ is a nonzero element;
		\item $ c=ab $ implies that $ a $ or $ b $ is a unit.
	\end{enumerate}
An element $ p $ of $ R $ is \textit{prime} provided that
\begin{enumerate}
	\item $ p $ is a nonzero nonunit;
	\item $ p|ab $ implies $ p|a $ or $ p|b $.
\end{enumerate}
\end{definition}
\textbf{[Incomplete]}
\begin{definition}
	An integral domain $ R $ is a \textit{unique factorization domain} provided that every nonzero nonunit element can be written as a unique finite product of irreducibles up to re arrangements and up to multiplication by units.
\end{definition}
\begin{definition}
	Let $ \Natural $ be the set of natural numbers and $ R $ a commutative ring. $ R $ is a \textit{Euclidean ring} if there is a function $ \varphi:R-{0}\to \Natural $ such that:
	\begin{enumerate}
		\item If $ a,b\in R $ and $ ab\neq 0 $, then $ \varphi(a)\leqslant \varphi(ab) $;
		\item If $ a,b\in R $ and $ b \neq 0 $, then there exist $ q,r\in R $ such that $ a=qb+r $ with $ r=0 $ or $ r \neq 0 $ and $ \varphi(r)<\varphi(b) $.
	\end{enumerate}
A Euclidean ring which is an integral domain is called a \textit{Euclidean domain}.
\end{definition}
\begin{theorem}
	Every Euclidean ring $ R $ is a principal ideal ring with identity. Consequently every Euclidean domain is a unique factorization domain.
\end{theorem}
\begin{Example}
	Let $ \Zahlen[i] $ be $ \{a+bi|a,b \in \Zahlen \} $. $ \Zahlen[i] $ is an integral domain called the domain of Gaussian integers. Clearly $ \Zahlen[i] $ is an Euclidean domain provided $ \varphi(a+bi)=a^2+b^2 $.
\end{Example}
\begin{definition}
	Let $ X $ be a nonempty subset of a commutative ring $ R $. An element $ d\in R $ is a \textit{greatest common divisor} of $ X $ provided:
	\begin{enumerate}
		\item $ d|a $ for all $ a \in X $;
		\item $ c|a $ for all $ a \in X $ implies that $ c|d $.
	\end{enumerate}
If $ R $ has an identity and $ 1_R $ is the greatest common divisor of $ X $, then elements of $ X $ are said to be \textit{relatively prime}.
\end{definition}
\begin{definition}
		Let $ X $ be a nonempty subset of a commutative ring $ R $. An element $ l\in R $ is a \textit{least common multiple} of $ X $ provided:
	\begin{enumerate}
		\item $ a|l $ for all $ a \in X $;
		\item $ a|f $ for all $ a \in X $ implies that $ l|f $.
	\end{enumerate}
\end{definition}
\begin{theorem}
	Let $ R $ be a Euclidean domain. Any two elements $ a $ and $ b $ in $ R $ have a greatest common divisor $ g $. Moreover, there exist $ s,t \in R $ such that
	\begin{equation}
		g=sa+tb \nonumber
	\end{equation}
\end{theorem}
\begin{lemma}
	If $ r_{i-1}=r_i q_{i+1}+r_{i+1} $, then $ GCD(r_{i-1},r_i)=GCD(r_{i},r_{i+1}) $.
\end{lemma}
\begin{theorem}
	$ \Zahlen_n $ is a field iff $ n $ is prime.
\end{theorem}
\begin{theorem}
	Let $ a $ be an element of the Euclidean ring $ R $. The quotient ring $ R/(a) $ is a field iff $ a $ is irreducible over $ R $.
\end{theorem}
\section{Rings of Quotients and Localization}
\begin{definition}
	A nonempty subset $ S $ of a ring $ R $ is \textit{multiplicative} if $ a,b\in S \Rightarrow ab\in S $.
\end{definition}
\begin{theorem}
	Let $ S $ be a multiplicative subset of a commutative ring $ R $. The relation defined on the set $ R \times S $ by
	\begin{equation}
		(r,s)\sim (r^\prime ,s^\prime)\Leftrightarrow s_1 (rs^\prime-r^\prime s)=0\quad \text{for some}\quad s_1 \in S\nonumber
	\end{equation}
	is an equivalence relation. Furthermore if $ R $ has no zero divisors and $ 0\notin S $, then
	\begin{equation}
		(r,s)\sim (r^\prime ,s^\prime)\Leftrightarrow  rs^\prime-r^\prime s=0\nonumber
	\end{equation}
\end{theorem}
$ (r,s) $ will be denoted as $ r/s $ from now on.
\begin{theorem}
	Let $ S $ be a multiplicative subset of a commutative ring $ R $ and let $ S^{-1}R $ be the set of equivalence classes of $ R\times S $ under the equivalence relation defined previously.
	\begin{enumerate}
		\item $ S^{-1}R $ is a commutative ring with identity, where addition and multiplication are defined similarly to the addition and multiplication of rationals.
		\item If $ R $ is a nonzero ring with no zero divisors and $ 0\notin S $, then $ S^{-1}R $ is an integral domain.
		\item If $ R $ is a nonzero ring with no zero divisors and $ S $ is the set of all nonzero elements in $ R $, then $ S^{-1}R $ is a field.
	\end{enumerate}
\end{theorem}
\begin{definition}
	$ S^{-1}R $ is called the \textit{ring of quotients} or \textit{ring of fractions} or \textit{quotient ring} of $ R $ by $ S $. If $ S $ is the nonzero elements of $ R $, $ S^{-1}R $ is called the \textit{quotient field of the integral domain $ R $}(as in the third statement of the previous theorem). More generally if $ R $ is any non-zero commutative ring and $ S $ is the non-empty set of all nonzero elements of $ R $ that are not zero divisors, $ S^{-1}R $ is called the \textit{complete}(or \textit{full}) \textit{ring of quotients}(or \textit{fractions}) of the ring $ R $.
\end{definition}
\begin{theorem}
	Let $ S $ be a multiplicative subset of a commutative ring $ R $.
	\begin{enumerate}
		\item The map $ \varphi_S:R \mapsto S^{-1}R $ given by $ r\mapsto rs/s $(for any $ s\in S $) is a well-defined homomorphism of rings such that $ \varphi_S(s) $ is a unit in $ S^{-1}R $ for every $ s\in S $.
		\item If $ 0\notin S $ and $ S $ contains no zero divisors, then $ \varphi_S $ is a monomorphism. In particular, any integral domain may be embedded in its quotient field.
		\item If $ R $ has an identity and $ S $ consists of units, then $ \varphi_S $ is an isomorphism. In particular, the complete ring of quotient, or the quotient field, of a field $ F $ is isomorphic to $ F $.
	\end{enumerate}
\end{theorem}
\begin{theorem}
	Let $ S $ be a multiplicative subset of a commutative ring $ R $ and let $ T $ be any commutative ring with identity. If $ f:R\to T $ is a homomorphism of rings such that $ f(s) $ is a unit in $ T $ for all $ s \in S $, then there exists a unique homomorphism of rings $ \bar{f}:S^{-1}R\to T $ such that $ \bar{f}\varphi_S=f $. The ring $ S^{-1}R $ is completely determined(up to isomorphism) by this property.
\end{theorem}
\begin{Corollary}
	Let $ R $ be an integral domain considered as a subring of its quotient field $ F $. If $ E $ is a field and $ f:R\to E $ a monomorphism of rings, then there is a unique monomorphism of fields $ \bar{f}:F \to E $ such that $ \bar{f}|R=f $. In particular any field $ E_1 $ containing $ R $ contains an isomorphic copy $ F_1 $ of $ F $ with $ R \subset F_1 \subset E_1 $.
\end{Corollary}
\begin{theorem}
	Let $ S $ be a multiplicative subset of a commutative ring $ R $.
	\begin{enumerate}
		\item If $ I $ is an ideal in $ R $, then $ S^{-1}I=\{a/s|a\in I;s \in S \} $ is an ideal is $ S^{-1}R $.
		\item If $ J $ is another ideal in $ R $, then
		\begin{align}
			S^{-1}(I+J)&=S^{-1}I+S^{-1}J\nonumber\\
			S^{-1}(IJ)&=(S^{-1}I)(S^{-1}J)\nonumber\\
			S^{-1}(I\cap J)&=S^{-1}I \cap S^{-1}J\nonumber
		\end{align}
	\end{enumerate}
\end{theorem}
\begin{definition}
	$ S^{-1}I $ is called the \textit{extension} of $ I $ in $ S^{-1}R $.
\end{definition}
\begin{theorem}
	Let $ S $ be a multiplicative subset of a commutative ring $ R $ with identity and let $ I $ be an ideal of $ R $. Then $ S^{-1}I=S^{-1}R $ iff $ S \cap I \neq \varnothing $.
\end{theorem}
\begin{definition}
	If $ J $ is an ideal in a ring of quotients $ S^{-1}R $, then $ \varphi_S^{-1}(J) $ is an ideal in $ R $ and is sometimes called the \textit{contraction} of $ J $ in $ R $.
\end{definition}
\textbf{Incomplete}

\section{Rings of Polynomials and Formal Power Series}
\begin{theorem}
	Let $ R $ be a ring and let $ R[x] $ denote the set of all sequences of elements of $ R $ $ (a_0,a_1,\cdots) $ such that $ a_i=0 $ for all but a finite number of indices $ i $.
	\begin{enumerate}
		\item $ R[x] $ is a ring with addition and multiplication defined similarly to the addition and multiplication of polynomials in $ \Real $.
		\item If $ R $ is commutative(resp. a ring with identity or a ring without zero divisors or an integral domain), then so is $ R[x] $.
		\item The map $ R\to R[x] $ given by $ r \mapsto (r,0,0,\cdots) $ is a monomorphism of rings.
	\end{enumerate}
\end{theorem}
\begin{definition}
	The ring $ R[x] $ is called the \textit{ring of polynomials} over $ R $.
\end{definition}
\begin{theorem}
	Let $ R $ be a ring with identity and denote by $ x $ the element $ (0,1_R,0,\cdots) $ of $ R[x] $.
	\begin{enumerate}
		\item $ x^n=(0,0,\cdots,0,1^R,0,\cdots) $, where $ 1_R $ is the $ (n+1) $st coordinate.
		\item If $ r\in R $, then for each $ n \geqslant 0 $, $ rx^n=x^nr=(0,\cdots,0,r,0,\cdots) $, where $ r $ is the $ (n+1) $st coordinate.
		\item For every nonzero polynomial $ f\in R[x] $ there exists an integer $ n $ and elements $ a_0,\cdots,a_n\in R $ such that $ f=a_0 x^0+a_1 x^1 + \cdots + a_n x^n $. The integer $ n $ and elements $ a_i $ are unique.
	\end{enumerate}
\end{theorem}
\begin{definition}
	If $ f=\sum_{i=0}^{n}a_i x^i \in R[x]$, then $ a_i $ are called the \textit{coefficients} of $ f $. $ a_0 $ is the constant term. Elements of $ R[x] $ whose only nonzero coordinates is the first one are called \textit{constant polynomials}. $ a_n $ is called the \textit{leading coefficient} of $ f $. If $ R $ has an identity and the leading coefficient of $ f $ is $ 1_R $ , $ f $ is said to be a \textit{monic polynomial}. The element $ x=(0,1_R,\cdots) $ is called an \textit{indeterminate}.
\end{definition}
\begin{theorem}
	Let $ R $ be a ring and denote by $ R[x_1,x\cdots,x_n] $ the set of all functions $ f:\Natural^n \to R $ such that $ f(u)\neq 0 $ for at most a finite number of elements $ u $ of $ \Natural^n $.
	\begin{enumerate}
		\item $ R[x_1,\cdots,x_n] $ is a ring with addition and multiplication defined by
		\begin{equation}
			(f+g)(u)=f(u)+g(u)\quad \text{and}\quad (fg)(u)=\sum_{v+w=n; v,w\in \Natural^n}f(v)g(w)\nonumber
		\end{equation}
		where $ f,g \in R[x_1,x\cdots,x_n] $ and $ u \in \Natural^n $.
		\item If $ R $ is commutative(resp. a ring with identity or a ring without zero divisors or an integral domain), then so is $ R[x_1,\cdots,x_n] $.
		\item The map $ R\to R[x_1,\cdots,x_n] $ given by $ r \mapsto f_r $, where $ f_r(0,0,\cdots,0)=r $ and $ f(u)=0 $ for all other $ u \in \Natural^n $, is a monomorphism of rings.
	\end{enumerate}
\end{theorem}
\begin{definition}
	The ring $ R[x_1,\cdots,x_n] $ is called the \textit{ring of polynomials in $ n $ indeterminates over $ R $}.
\end{definition}
\textbf{Incomplete}
\begin{Proposition}
	Let $ R $ be a ring and denote by $ R[[x]] $ the set of all sequences of elements of $ R $ $ (a_0,a_1,\cdots) $.
	\begin{enumerate}
		\item $ R[[x]] $ is a ring with component-wise addition and multiplication defined by 
		\begin{equation}
			(a_0,a_1,\cdots)(b_0,b_1,\cdots)=(c_0,c_1,\cdots)\nonumber
		\end{equation}
		where $ c_n =\sum_{k+j=n}a_k b_j$.
		\item $ R[x] $ is a subring of $ R[[x]] $ .
		\item If $ R $ is commutative(resp. a ring with identity or a ring without zero divisors or an integral domain), then so is $ R[[x]] $.
	\end{enumerate}
\end{Proposition}
\begin{definition}
	$ R[[x]] $ is called the \textit{ring of formal power series} over the ring $ R $.
\end{definition}
\textbf{Incomplete}
\section{Factorization in Polynomial Rings}
\begin{definition}
	Let $ R $ be a ring. The \textit{degree of a nonzero monomial} $ a x_1^{k_1}x_2^{k_2}\cdots x_n^{k_n} in R[x_1,\cdots,x_n]$ is the nonnegative integer $ k_1 + k_2 + \cdots + k_n $. The \textit{(total) degree of the polynomial $ f $}, denoted $ \deg f $, is the maximum of the degrees of the monomials $ a_i x_1^{k_{i_1}}x_2^{k_{i_2}}\cdots x_n^{k_{i_n}} $ such that $ a_i \neq 0 $. A polynomial which is a sum of monomials, each of which has degree $ k $, is said to be \textit{homogeneous of degree $ k $}. The \textit{degree of $ f $ in $ x_k $} is the degree of $ f $ considered as a polynomial in one indeterminate $ x_k $ over the ring $ R[x_1,\cdots,x_{k-1},x_{k+1},\cdots,x_n] $.
\end{definition}
\begin{theorem}
	Let $ R $ be a ring and $ f,g \in R[x_1,\cdots,x_n] $.
	\begin{enumerate}
		\item $ \deg(f+g) \leqslant \max (\deg f, \deg g)$.
		\item $ \deg (fg)\leqslant \deg f + \deg g $.
		\item If $ R $ has no zero divisors, $ \deg(fg)=\deg f+ \deg g $.
		\item If $ n=1 $ and the leading coefficient of $ f $ or $ g $ is not a zero divisor in $ R $( in particular, if it is a unit), then $ \deg(fg)=\deg f+ \deg g  $.
	\end{enumerate}
\end{theorem}
\begin{theorem}[The Division Algorithm]
	If $ R $ is a ring with identity and $ f,g\in R[x] $ are nonzero polynomials such that the leading coefficient of $ g $ is a unit in $ R $, there exist unique polynomials $ q,r\in R[x] $ such that
	\begin{equation}
		f=qg+r\qquad \text{and}\qquad \deg r < \deg g\nonumber
	\end{equation}
\end{theorem}
\begin{Corollary}[Remainder theorem]
	Let $ R $ be a ring with identity and
	\begin{equation}
		f(x)=\sum_{i=0}^{n}a_i x^i \in R[x]\nonumber
	\end{equation}
	For any $ c \in R $ there exists a unique $ q(x) \in R[x]$  such that $ f(x)=q(x)(x-c)+f(c) $.
\end{Corollary}
\begin{Corollary}
	If $ F $ is a field, then $ F[x] $ is a Euclidean domain, whence $ F[x] $ is a principal ideal domain and a unique factorization domain. The units in $ F[x] $ are precisely the nonzero constant polynomials.
\end{Corollary}
\begin{Corollary}
	$ (x-\alpha) $ is a factor of $ f(x) $ in $ F[x] $ iff $ f(\alpha)=0 $.
\end{Corollary}
\begin{Corollary}
	A polynomial of degree $ n $ in $ F[x] $ has at most $ n $ roots in $ F $.
\end{Corollary}
\begin{theorem}[Fundamental theorem of Algebra]
	If $ f(x) $ is a polynomial in $ \Complex[x] $ of positive degree, then $ f(x) $ has a root in $ \Complex $.
\end{theorem}
\begin{theorem}
\begin{enumerate}
	\item 	If $ z $ is a complex root of the real polynomial $ f(x)\in \Real[x] $, then its conjugate $ \overline{z} $ is also a root.
	\item If $ a,b,c \in \Quoziente $ and $ a+b\sqrt{c} $ is an irrational root of the rational polynomial $ f(x)\in \Quoziente[x] $, then $ a-b \sqrt{c} $ is also a root.
\end{enumerate}
\end{theorem}
\begin{theorem}
	\begin{enumerate}
		\item The irreducible polynomials in $ \Complex[x] $ are the polynomials of degree one.
		\item The irreducible polynomials in $ \Real[x] $ are the polynomials of degree $ 1 $ together with the polynomials of degree $ 2 $ of the form $ ax^2 + bx + c $, where $ b^2 < 4ac $.
	\end{enumerate}
\end{theorem}
\begin{theorem}
	Let $ p(x)=a_0+ a_1 x + \cdots +a_n x^n \in \Zahlen[x]$. If $ r/s $ is a rational root of $ p(x) $ and $ GCD(r,s)=1 $, then $ r|a_0 $ and $ s|a_n $.
\end{theorem}
\begin{lemma}[Gauss's lemma]
	Let $ P(x) =a_0 + a_1 x + \cdots +a_n x^n \in \Zahlen[x]$. If $ P(x) $ can be factored in $ \Quoziente[x] $ as $ P(x)=q(x)r(x) $, then $ P(x) $ can also be factored in $ \Zahlen[x] $.
\end{lemma}
\begin{theorem}[Eisenstein's Criterion]
	Let $ D $ be a unique factorization domain with quotient field $ F $. If $ f=\sum_{i=0}^{n}a_i x^i \in D[x] $, $ \deg f \geqslant 1 $ and $ p $ is an irreducible element of $ D $ such that
	\begin{equation}
		p\nmid a_n;\quad p|a_i\quad \text{for}\quad i=0,1,\cdots,n-1;\quad p^2 \nmid a_0\nonumber
	\end{equation}
	then $ f $ is irreducible in $ F[x] $. If $ f $ is primitive, then $ f $ is irreducible in $ D[x] $.
\end{theorem}
\begin{Corollary}
	Let $ f(x)= \sum_{i=0}^{n}a_i x^i \in \Zahlen[x] $. If, for some prime $ p $,
	\begin{enumerate}
		\item $ p|a_i,\quad i=0,1,\cdots,n-1 $;
		\item $ p \nmid a_n $;
		\item $ p^2 \nmid a_0 $.
	\end{enumerate}
then $ f(x) $ is irreducible over $ \Quoziente $.
\end{Corollary}
\begin{Example}
	For any prime $ p $ the polynomial $ \varphi(x) =x^{p-1}+x^{p-1}+\cdots + x + 1$ is irreducible over $ \Quoziente $. This polynomial is called a \textit{cyclotomic polynomial} and can be written $ \varphi(x)=\frac{(x^p-1)}{x-1} $.
\end{Example}
\begin{theorem}
	Let $ P $ be the ideal $ (p(x)) $ in the polynomial ring of the field $ F[x] $, in which $ p(x) $ has a positive degree. The different elements of $ F[x]/(p(x)) $ are those of the form
	\begin{equation}
		P+a_0+a_1 x+\cdots + a_{n-1}x^{n-1}\nonumber
	\end{equation}
	where $ a_i \in F $.
\end{theorem}
\chapter{Modules}
\section{Modules, Homomorphisms and Exact Sequences}
\section{Free Modules and Vector Spaces}
\section{Projective and Injective Modules}
\section{Hom and Duality}
\section{Tensor Products}
\section{Modules over a Principal Ideal Domain}
\section{Algebras}


\chapter{Fields and Galois Theory}
\section{Field Extensions}
\begin{definition}
	A field $ F $ is said to be an \textit{extension field} of $ K $(or simply an \textit{extension} of $ k $) provided that $ K $ is a subfield of $ F $.
\end{definition}
If $ F $ is an extension field of $ K $, $ F $ is a vector space over $ K $. The \textit{dimension} of the $ K $-vector space $ F $ will be denoted by $ [F:K] $. $ F $ is said to be a \textit{finite dimensional extension} or \textit{infinite dimensional extension} of $ K $ according as $ [F:K] $ is finite or infinite.
\begin{theorem}
	Let $ F $ be an extension field of $ E $ and $ E $ an extension field of $ K $. Then $ [F:K]=[F:E][E:K] $. $ [F:K] $ is finite iff $ [F:E] $ and $ [E:K] $ are finite.
\end{theorem}
\begin{Example}
	$ [\Complex:\Real]=2 $.
\end{Example}
\begin{Example}
	If $ p(x) $ is irreducible over the field $ F $, then $ K=F[x]/(p(x)) $ is an extension field of $ F $. Furthermore $ [K:F]=\deg p(x) $.
\end{Example}
\textbf{[Incomplete]}
\begin{definition}
	Let $ F $ be an extension of $ K $. An element $ u $ of $ F $ is said to be \textit{algebraic} over $ K $ if $ u $ is a root of some nonzero polynomial $ f\in K[x] $. Otherwise $ u $ is \textit{transcendental} over $ K $. $ F $ is called an \textit{algebraic extension} of $ K $ if every element of $ F $ is algebraic over $ K $. $ F $ is called a \textit{transcendental extension} if at least one element of $ F $ is transcendental over $ K $.
\end{definition}
\begin{Example}
	If $ K $ is a field, then $ K[x_1,\cdots,x_n] $ is an integral domain. The quotient field of $ K[x_1,\cdots,x_n] $ is denoted $ K(x_1,\cdots,x_n) $, which consists of all fractions of elements in $ K[x_1,\cdots,x_n] $. $ K(x_1,\cdots,x_n) $ is called the \textit{field of rational fractions} in $ x_1,\cdots,x_n $ over $ K $. Every element of $ K(x_1,\cdots,x_n) $ is transcendental over $ K $.
\end{Example}
\begin{theorem}
	If $ F $ is an extension field of $ K $ and $ u\in F $ is algebraic over $ K $, then
	\begin{enumerate}
		\item $ K(u)=K[u] $;
		\item $ K(u)\cong F[x]/(p(x)) $, where $ f\in K[x] $ is an irreducible monic polynomial of degree $ n\geqslant 1 $ uniquely determined by the conditions that $ f(u)=0 $ and $ g(u)=0 $($ g\in K[x] $) iff $ f $ divides $ g $;
		\item $ [K(u):K]=n $;
		\item $ \{1_K,u,u^2,\cdots,u^{n-1} \} $ is a basis of the vector space $ K(u) $ over $ K $;
	\end{enumerate}
\end{theorem}
\begin{lemma}
	Let $ p(x) $ be an irreducible polynomial over the field $ F $. Then $ F $ has a finite extension field $ K $ in which $ p(x) $ has a root.
\end{lemma}
\begin{theorem}
	If $ f(x) $ is any polynomial over the field $ F $, there is an extension field $ K $ of $ F $ over which $ f(x) $ splits into linear factors.
\end{theorem}
\begin{Proposition}
	If $ [K:F] =2$ where $ F \subset \Quoziente $, then $ K=F(\sqrt{\gamma}) $ for some $ \gamma \in F $.
\end{Proposition}
\begin{Proposition}
	If $ F $ is an finite extension of $ \Real $, then $ F $ is isomorphic to $ \Real $ or $ \Complex $.
\end{Proposition}
\begin{Example}
	$ [R:\Quoziente] $ is infinite.
\end{Example}
\section{The Fundamental theorem}
\section{Splitting Fields, Algebraic Closure and Normality}
\section{The Galois Group of a Polynomial}
\section{Finite Fields}
\begin{Proposition}
	The characteristic of an integral domain is either zero or prime.
\end{Proposition}
\begin{Corollary}
	If $ F $ is a finite field, then $ \Char F=p\neq 0 $ for some prime $ p $ and $ |F|=p^n $ for some integer $  n\geqslant 1 $.
\end{Corollary}
\begin{theorem}
	If $ F $ is a field and $ G $ is a finite subgroup of the multiplicative group of nonzero elements of $ F $, then $ G $ is a cyclic group.
\end{theorem}
\begin{Proposition}
	If the field $ F $ has prime characteristic $ p $, then $ F $ contains a subfield isomorphic to $ \Zahlen_p $. If the field $ F $ has zero characteristic, then $ F $ contains a subfield isomorphic to the rational numbers.
\end{Proposition}
\begin{definition}
	A finite field with $ p^m $ elements is called a \textit{Galois field} of order $ p^m $ and is denoted by $ GF(p^m) $. It can be shown that for a given prime $ p $ and positive integer $ m $, a Galois field $ GF(p^m) $ exists and that all fields of order $ p^m $ are isomorphic. Moreover
	\begin{equation}
		GF(p^m)=\Zahlen_p[x]/(q(x))\nonumber
	\end{equation}
	where $ q(x) $ is a degree $ m $ polynomial irreducible in $ \Zahlen_p[x] $.
\end{definition}
\begin{definition}
	Elements of a Galois field $ GF(p^m) $ can be written as
	\begin{equation}
		\{a_0+a_i\alpha+\cdots+a_{m-1}\alpha^{m-1}|a_i\in\Zahlen_p \}\nonumber
	\end{equation}
	where $ \alpha $ is a root of a polynomial $ q(x) $ of degree $ m $ irreducible over $ \Zahlen_p $. With judicious choice of $ \alpha $ the elements of $ GF(p^m) $ can be written as 
	\begin{equation}
		\{0,1,\alpha,\alpha^2,\alpha^3,\cdots,\alpha^{p^m-2} \}\quad \text{where}\quad \alpha^{p^m-1}=1\nonumber
	\end{equation}
	The element $ \alpha $ is called a \textit{primitive element} of $ GF(p^m) $. Equivalently a generator of the cyclic group $ (GF(q)^\ast,\cdot) $ is called a \textit{primitive element of $ GF(q) $ }.
\end{definition}
\begin{definition}
	An irreducible polynomial $ g(x) $ of degree $ m $ over $ \Zahlen_p $ is called a \textit{primitive polynomial} if $ g(x)|x^k-1 $ for $ k=p^m-1 $ and for no smaller $ k $.
\end{definition}
\begin{Proposition}
	The irreducible polynomial $ g(x)\in \Zahlen_p[x] $ is primitive iff $ x $ is a primitive element in $ \Zahlen_p[x]/(g(x)) =GF(p^m)$.
\end{Proposition}

\section{Separability}
\section{Cyclic Extensions}
\section{Cyclotomic Extensions}
\section{Radical Extensions}


\chapter{The Structure of Fields}
\section{Transcendence Bases}
\section{Linear Disjointness and Separability}

\chapter{Commutative Rings and Modules}
\section{Chain Conditions}
\section{Prime and Primary Ideals}
\section{Primary Decomposition}
\section{Noetherian Rings and Modules}
\section{Ring Extensions}
\section{Dedekind Domains}
\section{The Hilbert Nullstellensatz}


\chapter{The Structure of Rings}
\section{Simple and Primitive Rings}
\section{The Jacobson Radical}
\section{Semisimple Rings}
\section{The Prime Radical; Prime and Semiprime Rings}
\section{Algebras}
\section{Division Algebras}


\chapter{Categories}
\section{Functors and Natural Transformations}
\section{Adjoint Functors}
\section{Morphisms}

\chapter{Applications}	
\section{Euclidean Motions}
\begin{definition}
	An \textit{isometry} of $ \Real^n $ is a transformation $ f:\Real^n \to \Real^n $ which is continuous and a symmetry(bijection). An isometry preserves the inner product(dot product on $ \Real^n $), thus preserves length and angle. Isometries are ``Rigid Motions'' os $ \Real^n $. The group of all isometries of $ \Real^n $ is called the \textit{Euclidean Group of $ \Real^n $} and is denoted $ E(n) $.
\end{definition}
\begin{definition}
	The \textit{orthogonal group}, denoted $ O(n) $, is a subgroup of $ E(n) $ consisting of all linear transformations that preserves inner products, which says that it is a group of matrices. The fact that these matrices preserve length implies that they have orthonormal columns: each column has unit length and two different columns have their dot product equal to zero.
\end{definition}
\begin{lemma}
	If $ \matr{A}\in O(n) $, then $ \matr{A}\times \matr{A}^T=I $.
\end{lemma}
\begin{Proposition}
	$ O(n)=\{\matr{A} \in E(n),\matr{A}(\vec{0})=(\vec{0}) \} $(If an isometry maps $ \vec{0} $ to $ \vec{0} $, then it is a linear transformation).
\end{Proposition}
\begin{definition}
	Let $ T(n) $, the \textit{group of translations}, be the subgroup of $ E(n) $ such that if $ \alpha(\vec{v})=\vec{v}-\vec{v}_0 $ for some fixed $ \vec{v_0}\in \Real^n $.
\end{definition}
\begin{Proposition}
	There is a epimorphism from $ E(n) $ to $ O(n) $, defined by $ \varphi:E(n) \to O(n)$ and $ \varphi(\alpha)(\vec{v})=\alpha(\vec{v})-\alpha(\vec{0}) $, whose kernel is $ T(n) $.
\end{Proposition}
\begin{Corollary}
	$ T(n) $ is a normal subgroup of $ E(n) $ and $ E(n)/T(n)\cong O(n) $.
\end{Corollary}
\begin{Proposition}
	Every finite subgroup $ G $ of $ E(n) $ fixes at least one point, i.e. there exists a vector $ \vec{v}\in\Real^n $ such that $ g(\vec{v})=\vec{v} $ for all $ g\in G $.
\end{Proposition}
\section{Matrix Groups}
\begin{definition}
	Let $ GL(n,\Real) $ denote the set of all nonsingular matrices with real entries and $ GL(n,\Complex) $ be the set of all nonsingular matrices with complex entries. These groups are called \textit{General Linear Groups}(and hence the derivation of the abbreviations).
\end{definition}
\begin{Proposition}
	Let $ G $ be a finite subgroup of $ GL(n,\Real) $(or $ GL(n,\Complex) $). Then $ \det (g) $ is a root of unity for any $ g\in G $.
\end{Proposition}
The function $\det: O(n) \to \{1,-1\}\cong \Zahlen_2 $, mapping a matrix to its determinant, is an epimorphism between groups.
\begin{definition}
	$ SO(n) $, or the \textit{Special Orthogonal Group}, is the subgroup of $ O(n) $ such that $ \det(a)=1 $ for $ a\in SO(n) $(an equivalent way is that $ SO(n) $ is the kernel of $ \det :O(n) \to \{1,-1\}\cong \Zahlen_2$ defined above). It is the group of proper rotations in $ \Real^n $.
\end{definition}
\begin{definition}
	Let $ U(n) $ be the subgroup of $ GL(n,\Complex) $ of complex unitary transformations
	\begin{equation}
		U(n)=\{\matr{A}\in GL(n,\Complex)|\angbracket{\matr{A}(\vec{u}),\matr{A}(\vec{v})}=\angbracket{\vec{u},\vec{v}} \}\nonumber
	\end{equation}
	where $ \angbracket{(x_1,\cdots,x_n),(y_1,\cdots,y_n)}=x_1 \bar{y_1}+\cdots+x_n \bar{y_n} $, $ \bar{y} $ denotes the complex conjugate of $ y$. $ \matr{A}\in U(n) $ iff $ \matr{A}\overline{\matr{A}^T}=\matr{I} $. The length of $ \det(\matr{A}) $ is an element of the circle group $ S^1\subset \Complex $. The function $ \det:U(n)\to S^1 $ is in fact an epimorphism. The kernel of this epimorphism is those matrices having determinant $ 1 $, which composed of the set of \textit{Special Unitary Matrices} $ SU(n) $, a subgroup of $ U(n) $.
\end{definition}
\begin{definition}
	The \textit{representation} of a group $ G $ is a homomorphism from $ G $ to $ GL(n,K) $, where $ K=\Real $ or $ \Complex $. Every $ g \in G $ is represented as a matrix acting on $ \Real^n $ or $ \Complex^n $. The homomorphism is called \textit{faithful} if it is injective.
\end{definition}
\section{The $ 2 \times 2 $ Matrix Group}
\begin{theorem}
	$ O(2) $ consists of matrices of the form
	\begin{equation}
		\begin{pmatrix}
		\cos \theta &-\sin \theta \\
		\sin \theta &\cos \theta
		\end{pmatrix}\qquad \text{and}\qquad
		\begin{pmatrix}
		\cos \theta &\sin \theta \\
		\sin \theta &-\cos \theta
		\end{pmatrix} \nonumber
	\end{equation}
	where $0 \leqslant \theta \leqslant 2\pi $. Moreover $ SO(2)= \begin{pmatrix}
	\cos \theta &-\sin \theta \\
	\sin \theta &\cos \theta
	\end{pmatrix},0 \leqslant \theta \leqslant 2\pi $, and $ SO(2)\cong S^1 $.
\end{theorem}
\begin{theorem}
	If $ G $ is a finite subgroup of $ SO(2) $, then $ G \cong \Zahlen_n $ for some $ n $.
\end{theorem}
\begin{theorem}
	If $ G $ is a finite subgroup of $ O(2) $, then $ G \cong \Zahlen_n $ or $ G \cong D_n $ for some $ n $.
\end{theorem}
\section{Rotation of Regular Solids}
\begin{theorem}
	If $ \matr{A}\in SO(3) $, then $ \matr{A} $ has a fixed axis(a line through the origin) and $ \matr{A} $ is just rotation about the axis.
\end{theorem}
\begin{theorem}
	The group $ G $ of proper rotations of the tetrahedron is isomorphic to $ A_4 $.
\end{theorem}
\begin{theorem}
	The group of proper rotations of a cube is isomorphic to $ S_4 $.
\end{theorem}
\begin{Example}
	The group of proper rotations of a regular dodecahedron and the group of proper rotations of a regular icosahedron are both isomorphic to $ A_5 $.
\end{Example}
\begin{Example}
	The group of proper rotations of a octahedron is isomorphic to $ S_4 $.
\end{Example}
\section{Finite Rotation Groups and Crystallographic Groups}
\begin{theorem}
	Any finite subgroup of $ SO(3) $ is isomorphic to one of the following: $ \Zahlen_n $($ n\geqslant 1 $), $ D_n $($ n\geqslant 2 $), $ A_4 $, $ S_4 $, $ A_5 $.
\end{theorem}
\begin{definition}
	An \textit{ideal crystallite lattice $ L $} is a subset of $ \Real^3 $ of the form
	\begin{equation}
		L=\{n_1 \vec{v}_1 +n_2 \vec{v}_2 + n_3 \vec{v}_3|n_i \in \Zahlen \}\nonumber
	\end{equation}
	where $ \vec{v}_i $ is a fixed basis of $ \Real^3 $.
\end{definition}
\begin{definition}
	A subgroup of $ SO(3) $ or $ O(3) $ that leaves a crystallite lattice invariant is called a \textit{crystallographic point group}.
\end{definition}
\section{Polya-Burnside Method}
\begin{theorem}[Burnside]
	Let $ G $ be a finite group acting on a finite set $ X $. For $ g\in G $ let $ \Fix g $ be the set $ \{x\in X| g(x)=x \} $. If $ N $ is the number of orbits of $ X $ under $ G $, then
	\begin{equation}
	N=	\frac{1}{|G|}\sum_{g\in G}|\Fix g|\nonumber
	\end{equation}
\end{theorem}
	
	
	
\end{document}