\section{Homomorphisms and Subgroups}
\begin{definition}
	Let $ G $ and $ H $ be semigroups. A function $ f:G \to H $ is a \textit{homomorphism} provided
	\begin{equation}
		f(ab)=f(a)f(b) \text{ for all }a,b \in G\nonumber
	\end{equation}
	If $ f $ is injective as a map of sets, $ f $ is said to be a \textit{monomorphism}. If $ f $ is surjective, $ f $ is called an \textit{epimorphism}. If $ f $ is bijective, $ f $ is called an \textit{isomorphism}. In this case $ G $ and $ H $ are said to be \textit{isomorphic} (written $ G \cong H $). A homomorphism $ f:G \to G $ is called an \textit{endomorphism} of $ G $ and an isomorphism $ f:G \to G $ is called an \textit{automorphism} of $ G $.
\end{definition}
\begin{definition}
	Let $ f:G \to H $ be a homomorphism of groups. The \textit{kernel} of $ f $(denoted $ \Ker{f} $) is $ \{a\in G | f(a) = e \in H  \} $. If $ A $ is a subset of $ G $, then $ f(A)=\{b \in H | b= f(a) \text{ for some }a \in A \} $  is the \textit{image of $ A $}. $ f(G) $ is called the \textit{image of $ f $} and denoted $ \Im{f} $. If $ B $ is a subset of $ H $, $ f^{-1}(B)=\{a \in G |f(a) \in B  \} $ is the \textit{inverse image} of $ B $.
\end{definition}
\begin{theorem}
	Let $ f:G \to H $ be a homomorphism of groups. Then
	\begin{itemize}
		\item $ f $ is a monomorphism iff $ \Ker{f}=\{e\} $.
		\item $ f $ is an isomorphism iff there is a homomorphism $ f^{-1}:H \to G $ such that $ f f^{-1}=1_H $ and $ f^{-1}f=1_G $.
	\end{itemize}
\end{theorem}
\begin{definition}
	Let $ G $ be a semigroup and $ H $ a nonempty subset of it. If for every $ a,b \in H $ we have $ ab \in H $, we say that $ H $ is \textit{closed} under the product in $ G $. This is the same as saying that the binary operation on $ G $, when restricted to $ H $, is a binary operation on $ H $.
\end{definition}
\begin{definition}
	Let $ G $ be a group and $ H $ a nonempty subset that is closed under the product in $ G $. If $ H $ is itself a group under the product in $ G $, then $ H $ is said to be a \textit{subgroup} of $ G $, denoted $ H<G $.
\end{definition}
\begin{definition}
	If a subgroup $ H $ is not $ G $ itself or the \textit{trivial subgroup}, which consists only of the identity element, is called a \textit{proper subgroup}.
\end{definition}
\begin{theorem}
	Let $ H $ be a nonempty subset of a group $ G $. Then $ H $ is a subgroup of $ G $ iff $ a b^{-1}\in H $ for all $ a,b \in H $.
\end{theorem}
\begin{Corollary}
	If $ G $ is a group and $ \{H_i|i \in I \} $ is a nonempty family of subgroups, then $ \underset{i \in I}{\bigcap}H_i $ is a subgroup of $ G $.
\end{Corollary}


\begin{definition}
	Let $ G $ be a group and $ X $ a subset of $ G $. Let $ \{H_i|i \in I \} $ be the family of all subgroups of $ G $ which contain $ X $. Then $ \underset{i \in I}{\bigcap}H_i $ is called the \textit{subgroup of $ G $ generated by the set $ X $} and denoted $ \angbracket{X}$. The elements of $ X $ are the \textit{generators} of $ \angbracket{X} $. If $ G=\angbracket{a_1,\cdots,a_n} ,(a_i \in G)$, $ G $ is said to be finitely generated. If $ a \in G $, the subgroup $ \angbracket{a} $ is called the \textit{cyclic (sub)group} generated by $ a $.
\end{definition}

\begin{theorem}
	If $ G $ is a group and $ X $ a nonempty subset of $ G $, then the subgroup $ \angbracket{X} $ generated by $ X $ consists of all finite products $ a_1^{n_1} a_2^{n_2}\cdots a_t^{n_t}(a_i \in X;n_i \in \Zahlen)$. In particular for every $ a \in G $, $ \angbracket{a}=\{a^n|n\in \Zahlen \} $.
\end{theorem}

\begin{definition}
	The subgroup $ \angbracket{\underset{i \in I}{\bigcap}H_i } $ generated by the set $ \underset{i \in I}{\bigcap}H_i  $ is called the \textit{subgroup generated by the groups $ \{H_i | i \in I \} $}. If $ H $ and $ K $ are subgroups, the subgroup $ \angbracket{H \cup K} $ generated by $ H $ and $ K $ is called the \textit{join} of $ H $ and $ K $ and is denoted $ H \lor K $.
\end{definition}

